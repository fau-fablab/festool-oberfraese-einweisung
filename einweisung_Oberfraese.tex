\newcommand{\basedir}{fablab-document}
\documentclass{\basedir/fablab-document}

\usepackage{minitoc} % Inhaltsübersicht je Section
% \usepackage{fancybox} %ovale Boxen für Knöpfe - nicht mehr benötigt
\usepackage{amssymb} % Symbole für Knöpfe
% \usepackage{subfigure,caption}
\usepackage{eurosym}
\usepackage{tabularx} % Tabellen mit bestimmtem Breitenverhältnis der Spalten
\usepackage{wrapfig} % Textumlauf um Bilder
\usepackage{todonotes}

\renewcommand{\texteuro}{\euro}

\linespread{1.2}

\date{Februar 2016}
\author{Philipp Hörauf, bearbeitet von Snifftorg}
\title{Einweisung Kreissäge}

\begin{document}
 % Hinweise an Package minitoc, doch bitte irgendwas zu generieren - wird für späteres \secttoc benötigt
\dosecttoc
\faketableofcontents
\mtcsettitle{secttoc}{Arbeitsschritte}
\mtcsettitlefont{secttoc}{\large \sffamily \bfseries}
\mtcsetfont{secttoc}{subsection}{\sffamily}
% \mtcset
% hier geht das eigentliche Dokument los

\color{red}
\hrule
\begin{center}
\large{Achtung! Einweisung ist noch in Arbeit!}
\vspace{0.1cm}
\end{center}
\hrule
\color{black}

\section[Allgemeine Sicherheitshinweise]{Allgemeine Sicherheitshinweise}
\begin{itemize}
\item Hände bei aktiver Oberfräse immer vom Fräsbereich und dem Fräser fernhalten.
\item Die Oberfräse immer mit beiden Händen halten, entweder an Drehknopf, Handgriff und/oder den elektrisch isolierten Teilen des Gehäuses. Wenn beide Hände die Oberfräse halten, können sie vom Fräser nicht verletzt werden.
\item Nicht unter das Werkstück greifen.
\item Die Frästiefe an die Dicke des Werkstückes anpassen.
\item Netzkabel aus dem Eingriffsbereich des Fräsers halten. Die Metallteile des Elektrowerkzeugs könnten sonst unter Spannung stehen und einen elektrischen Schlag verursachen.
\item Das zu fräsende Werkstück niemals in der Hand, über dem Bein halten oder von Dritten halten lassen. Das Werkstück stabil und sicher fixieren. Es ist wichtig das Werkstück gut zu befestigen, um die Gefahr bei Körperkontakt, Klemmen des Fräsers oder Verlust der Kontrolle zu minimieren.
\item Werkzeug fest einspannen - dazu Schlüssel aus der Maschinenkiste verwenden
\item Nur Fräser der richtigen Größe verwenden. Fräser, die nicht zur Maschine passen, können zu Verlust der Kontrolle, schweren Verletzungen oder Beschädigung der Maschine führen.
\item Immer mit Absaugung fräsen.
\item Immer im Gegenlauf fräsen
\item Korrekte Drehzahl für jeweiliges Werkzeug und Werkstück vorwählen -> siehe Tabelle in der Anleitung, Seite 7
\item Niemals beschädigte Fräser verwenden. Es besteht die Gefahr des Festfressens oder Abreissens des Fräsers. Als beschädigt gelten Fräser dann, wenn sie Risse, Ausbrüche oder Verformungen zeigen, oder schlichtweg abgenutzt und stumpf sind.
\item Vor Werkzeugwechsel Maschine vom Stromnetz trennen, z.B. Stecker ziehen.
\item Nach dem Fräsen von Alu Maschine gründlichst reinigen und aussaugen.
\item Beim Fräsen immer einen Anschlag oder eine Führung verwenden. Dies verbessert die Genauigkeit und verringert die Gefahr, dass der Fräser verklemmt/sich festfrisst. Führung oder Anschlag bestimmungsgemäß fixieren.
\item Im Arbeitsbereich dürfen sich keine Unbeteiligten befinden
\end{itemize}


\section{persönliche Schutzausrüstung}

\begin{itemize}
\item Gehörschutz
\item Schutzbrille bei Bearbeitung von Alu, Faserwerkstoffen und Gipskarton
\item Staubmaske bei stauberzeugenden Arbeiten 
\item Schutzhandschuhe beim Bearbeiten rauher Materialien und beim Werkzeugwechsel.
\end{itemize}

\section{Bestimmungsgemäße Verwendung}
Die Oberfräse ist ein nützliches Gerät zur Bearbeitung von Kanten, zum Versäubern und auch zur kreativen Gestaltung (z.B. Einfräsen von Schrift oder Löchern in Plattenmaterial). Sie birgt, korrekt angewendet, nur ein geringes Verletzungsrisiko. Dennoch ist es wichtig, die folgenden Anweisungen genau zu beachten, damit die Oberfräse nicht zu Gesundheitsschäden führt oder beschädigt wird.\\
Es dürfen mit der Maschine und den beiligenden Fräsern Holz, Kunststoff und Aluminium bearbeitet werden. Bei der Verwendung der Maschine muss \textbf{IMMER} der Festool-Staubsauger verwendet werden. Dabei ist zu beachten, dass die Oberfräse an der Steckdose 1 des Saugers angeschlossen wird und der Sauger auf \enquote{AUTO} steht. Sauger mindestends auf halbe Kraft, besser auf Vollgas stellen.\\
\textbf{WICHTIG: Die Oberfräse darf ausschließlich von eingewiesene Personen verwendet werden.}


\subsection{Aluminiumbearbeitung}
Bei der Bearbeitung von Aluminium sind aus Sicherheitsgründen folgende besonderen Maßnahmen einzuhalten:
\begin{itemize}
\item Maschine an den Festool-Absaugwagen anschließen.
\item Maschine nach der Arbeit von Staubablagerungen im Motorgehäuse reinigen.
\item Nur geeignete Fräser verwenden.
\end{itemize}


\section{Inbetriebnahme}
Maschine vor dem Anschließen und Lösen der Netzanschlussleitung stets ausschalten! Anschließen und Lösen der Netzanschlussleitung siehe Bild [todo].
Dazu Ein-/Ausschalter (drücken= Ein / loslassen = AUS) und kontrollieren, dass die Einschaltsperre (kleiner Knopf am Gasgriff) herausspringt.

\section{Einstellungen}
\subsection{Funktionen der Maschine}
\begin{itemize}
\item Sanftanlauf: Der elektronisch geregelte Sanftanlauf sorgt für ruckfreien Anlauf des Elektrowerkzeugs.
\item Konstante Drehzahl: Die Motordrehzahl wird elektronisch konstant gehalten. Dadurch wird auch bei Belastung eine gleichbleibende Schnittgeschwindigkeit erreicht.
\item Drehzahlregelung: Die Drehzahl lässt sich mit dem Stellrad stufenlos einstellen. Dadurch können die Schnittgeschwindigkeit der jeweiligen Oberfläche optimal angepasst werden 
\item Temperatursicherung: Bei zu hoher Motortemperatur werden Stromzufuhr und Drehzahl reduziert. Die Maschine läuft nur noch mit verringerter Leistung, um eine rasche Abkühlung durch die Motorlüftung zu ermöglichen. Wenn die Übertemperatur andauert, schaltet die Maschine nach ca. 40 sec komplett ab. Erst nach Abkühlung des Motors ist ein erneutes Einschalten möglich.
\item Strombegrenzung: Die Strombegrenzung verhindert bei extremer Überlastung eine zu hohe Stromaufnahme. Dies kann zu einer Verringerung der Motordrehzahl führen. Nach Entlastung läuft der Motor sofort wieder an.
\item Bremse: Die OF 1010 EBQ besitzt eine elektronische Bremse. Nach dem Ausschalten wird der Fräser in ca. 1 Sekunde elektronisch zum Stillstand abgebremst.
\end{itemize}

\subsection{Frästiefe einstellen}
TODO siehe Anleitung

\subsection{Schnittwinkel einstellen}
TODO siehe Anleitung

\subsubsection{Fräser wechseln}
TODO siehe Anleitung

\section{Arbeiten mit der Maschine}
TODO siehe Anleitung

\section{Arbeiten mit Anschlagschiene}
TODO siehe Anleitung

\newpage
\section{Quellen}
%\begin{itemize}
%\item Festool \glqq Originalbetriebsanleitung\grqq TS 55 REBQ unter 
%\url{http://www.etracker.de/lnkcnt.php?et=6hsNGE&url=https\%3a\%2f\%2fassets.festool.com\%2fmedia\%2f706758_002_ts55rebq.zip&lnkname=Bedienungsanleitung+TS+55}
%\end{itemize}

\end{document}
